\documentclass[letterpaper,9pt]{book}

\usepackage{times}
\usepackage{lipsum}
%\usepackage[landscape]{geometry}
\setlength{\textheight}{419pt} \setlength{\textwidth}{289pt}
\setlength{\oddsidemargin}{72pt} \setlength{\evensidemargin}{108pt}
\setlength{\topmargin}{55.9pt} \setlength{\footskip}{27.5pt}
\setlength{\headheight}{14.6pt} \setlength{\headsep}{19.9pt}
\usepackage[print, 1to1]{booklet}
\setpdftargetpages

\title{A Daily Reader}

\begin{document}

\maketitle

\chapter{January 1, 2019}

The Cosmos is all that is, or ever was, or ever will be. Our contemplations of the Cosmos stir us. There's a tingling in the spine, a catch in the voice, a faint sensation, as of a distant memory, of falling from a great height. We know we are approaching the grandest of mysteries.

The size and age of the Cosmos are beyond ordinary human understanding. Lost somewhere between immensity and eternity is our tiny planetary home, the Earth. For the first time we have the power to decide the fate of our planet and ourselves. This is a time of great danger. But, our species is young and curious and brave and it shows much promise. In the last few millennia, we have made the most astonishing and unexpected discoveries about the Cosmos and our place within it. I believe our future depends powerfully on how well we understand this Cosmos, in which we float like a mote of dust in the morning sky.

We're about to begin a journey through the Cosmos. We'll encounter galaxies and suns, and planets, life and consciousness, coming into being, evolving and perishing; Worlds of ice and stars of diamond; atoms as massive suns, universes smaller than atoms. But it's also a story of our own planet, and the plants and animals that share it with us. And it's a story about us; how we achieved our present understanding of the Cosmos, how the Cosmos has shaped our evolution and our culture, what our fate may be.

We wish to pursue the truth no matter where it leads. But to find the truth we need imagination and skepticism both. We will not be afraid to speculate, but we will be careful to distinguish speculation from fact. The Cosmos is full beyond measure of elegant truths, of exquisite interrelationships, of the awesome machinery of nature.

The surface of the Earth is the shore of the cosmic ocean. On this shore, we have learned most of what we know. Recently, we have waded a little way out into the sea, maybe ankle deep, and the water seems inviting. The ocean calls to us. Some part of our being knows this is from where we came. We long to return. And we can because the Cosmos is also within us. We are made of star stuff. We are a way for the Cosmos to know itself.

The journey for each of us begins here. We're going to explore the Cosmos in a ship of the imagination, unfettered by ordinary limits on speed and size, drawn by the music of cosmic harmonies, that can take us anywhere in space and time. Perfect as a snowflake, organic as a dandelion seed, it can carry us to worlds of dreams and worlds of fact. Come with me.

\chapter{Chapter}
\lipsum[1-20]

\end{document}